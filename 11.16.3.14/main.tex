\let\negmedspace\undefined
\let\negthickspace\undefined
\documentclass[journal]{IEEEtran}
\usepackage[a5paper, margin=10mm, onecolumn]{geometry}
\usepackage{lmodern} % Ensure lmodern is loaded for pdflatex
\usepackage{tfrupee} % Include tfrupee package

\setlength{\headheight}{1cm} % Set the height of the header box
\setlength{\headsep}{0mm}     % Set the distance between the header box and the top of the text

\usepackage{gvv-book}
\usepackage{gvv}
\usepackage{cite}
\usepackage{amsmath,amssymb,amsfonts,amsthm}
\usepackage{algorithmic}
\usepackage{graphicx}
\usepackage{textcomp}
\usepackage{xcolor}
\usepackage{txfonts}
\usepackage{listings}
\usepackage{enumitem}
\usepackage{mathtools}
\usepackage{gensymb}
\usepackage{comment}
\usepackage[breaklinks=true]{hyperref}
\usepackage{tkz-euclide} 
\usepackage{listings}                                      
\def\inputGnumericTable{}                                 
\usepackage[latin1]{inputenc}                                
\usepackage{color}                                            
\usepackage{array}                                            
\usepackage{longtable}
\usepackage{multicol}
\usepackage{calc}                                             
\usepackage{multirow}                                         
\usepackage{hhline}                                           
\usepackage{ifthen}                                           
\usepackage{lscape}
\begin{document}

\bibliographystyle{IEEEtran}
\vspace{3cm}



\title{10.3.3.3.1}
\author{EE24BTECH11009 - Mokshith kumar}
% \maketitle
% \newpage
% \bigskip
{\let\newpage\relax\maketitle}

\renewcommand{\thefigure}{\theenumi}
\renewcommand{\thetable}{\theenumi}
\setlength{\intextsep}{10pt} % Space between text and floats


\numberwithin{equation}{enumi}
\numberwithin{figure}{enumi}
\renewcommand{\thetable}{\theenumi}

\textbf{Question:}
Given $p\brak{a}=\frac{3}{5}$ and $p\brak{b}=\frac{1}{5}$. Find $p\brak(a or b)$, if $a$ and $b$ are mutually exclusive events.\\
\textbf{solution:}
Given the events \( A \) and \( B \), where \( P(A) = \frac{3}{5} \) and \( P(B) = \frac{1}{5} \), and it is stated that \( A \) and \( B \) are mutually exclusive events, we are asked to find \( P(A \cup B) \), i.e., the probability of \( A \) or \( B \) occurring.\\
\begin{center}
\begin{tabular}{|c|c|}
    \hline
    $p\brak{A}$ & $\frac{3}{5}$ \\
    \hline
    $p\brak{B}$ & $\frac{1}{5}$\\
    \hline
\end{tabular}
\end{center}
Since \( A \) and \( B \) are mutually exclusive, it means that the events cannot occur simultaneously.\\
\[
P(AB)=0
\]
 For any two boolean variables A and B,
\begin{center}
\begin{align}
	\because A + A^\prime &= 1 \\
	 AB + A^\prime B &= B \label{2} \\
	 \implies \pr{AB} + \pr{A^\prime B} &= \pr{B} \label{3} \\
	 \because B + B^\prime &= 1 \\
	 AB + AB^\prime &= A \label{5}\\
	 \implies \pr{AB} + \pr{AB^\prime} &= \pr{A} \label{6} \\
	 \text{adding } \eqref{2} \text{ and } \eqref{5} \\
	 A + B &= AB + AB + AB^\prime + A^\prime B  \\
	 A + B &= AB + AB^\prime + A^\prime B \\ 
	 \pr{A + B} &= \pr{AB} + \pr{AB^\prime} + \pr{A^\prime B} \label{10}\\
	 \text{Adding \eqref{3},\eqref{6} and \eqref{10} and cancelling same terms } \\
	 \pr{AB} &= \pr{A} + \pr{B} - \pr{A + B}
\end{align}
\end{center}
From above Boolean Logic, the probability of \( A + B \) is given by the formula:

\[
P(A+B) = P(A) + P(B) - P(AB)
\]

Substitute the given values of \( P(A) \) and \( P(B) \):

\[
P(A + B) = \frac{3}{5} + \frac{1}{5}
\]

Simplifying the sum:

\[
P(A + B) = \frac{3 + 1}{5} = \frac{4}{5}
\]

Thus, the probability of \( A + B \) is:

\[
P(A \cup B) = \frac{4}{5}
\]

\end{document}